\documentclass{article}
\usepackage{graphicx}
\graphicspath{ {/home/users/jeem9157/} }

\begin{document} 

\title{Highly Eccentric Exoplanets Trapped in Mean-Motion Resonances}
\author{Jeremy Emmett}
\maketitle
	Over 500 known exoplanet systems contain at least two planets, and in many, mutual gravitational interactions
between planets can occur. When two bodies exert a mutual gravitational force on each other in a periodic 
manner, their orbital parameters can be gradually altered until a stable mean motion resonance is achieved. 
Mean motion resonance occurs when the ratio of the mean motion of two bodies approaches a small integer value, 
where the mean motion of a body is the time-averaged angular velocity over an entire orbit. If the eccentricity 
of a planet is sufficiently large, however, close encounters between planets occur leading to strong gravitational 
perturbations and thus instabilities in the orbits of the planetary system. This paper explores possible scenarios in
which a planet in a two planet can have a highly elliptical, yet stable, orbit in mean motion resonance with a second
planet.
\par
A two planet system was modeled with a simple planar three body problem, consisting of a massive star, and two 
planets of different masses. The evolving dynamics of the sytem were analyzed by plotting the eccentricity of the 
outermost planet to that of the innermost planet. In a simulation which approximates the mass of either planet as that
of Jupiter (0.001 x solar) and Saturn (0.0003 x solar), stable, resonant 2/1 orbits were achieved for eccentricity ratios 
close to 1.5 and 0.5. Please see Figure \ref{fig:figure1}. The model was then applied to a known exoplanet system with high eccentricity planetary orbits, 
HD 82943, using appropriate measured orbital parameters. The system was found to be in a stable, mean motion resonance,
and is predicted to remain such for many billions of years.
\begin{figure}
\label{figure1}
\includegraphics[scale=0.8]{e1e22}
\caption{}
\end{figure}

\bibliography{bib1}{}
\bibliographystyle{plain}

\end{document}
